%% genellipsis.dtx
%% Copyright 2017 F. Pijcke
%
% This work may be distributed and/or modified under the
% conditions of the LaTeX Project Public License, either version 1.3
% of this license or (at your option) any later version.
% The latest version of this license is in
%   http://www.latex-project.org/lppl.txt
% and version 1.3 or later is part of all distributions of LaTeX
% version 2005/12/01 or later.
%
% This work has the LPPL maintenance status `maintained'.
% 
% The Current Maintainer of this work is F. Pijcke.
%
% This work consists of the files genellipsis.sty and genellipsis.tex

\documentclass{ltxdoc}

\usepackage{genellipsis}
\usepackage{booktabs}

\makeatletter
\newcommand{\customtt}{\begingroup\catcode`\#=12\relax\@customtt}
\newcommand{\@customtt}[1]{\endgroup\texttt{\detokenize{#1}}}
\makeatother

\newcommand{\example}[1]{
  \begin{center}
    \begin{tabular}{p{0.42\textwidth} p{0.42\textwidth}}
      \toprule
      \customtt{#1} & #1 \\
      \bottomrule
    \end{tabular}
  \end{center}
}

\title{The \textsf{genellipsis} package}
\author{Fabian Pijcke}
\date{2017/03/01 --- v0.1}

\begin{document}
\DeleteShortVerb{\|} % ltxdoc interferes with genellipsis.

\maketitle

\begin{abstract}
  This package addresses the need to regularly type ellipses. The default
  configuration produces ellipses such as $\lip{1, 2,, n}$, including
  \texttt{\textbackslash linebreak}s after each comma and is as simple to use
  as \texttt{\textbackslash lip\{1, 2,, n\}}. An useful feature is that
  \texttt{\textbackslash lip} takes an optional command to customize each value
  in the ellipsis.
\end{abstract}

\section{Introduction}

Documents about mathematics often make use of set extensions such as $\{\lip{1,
2,, n}\}$, or maybe $\{\lip |\mathbf{h}(A_#1^2)| {1, 2,, n}\}$. To produce such
sets the following commands are typically used:

\begin{verbatim}
$\{1, 2, \dots, n\}$
\end{verbatim}

This formulation suffers from the drawback that it is unbreakable, resulting in
frequent Overful \texttt{\textbackslash hbox}es problems. One way to circumvent
this is to add \texttt{\textbackslash allowbreak} commands after each comma,
which is not really practical.

The \texttt{\textbackslash lip} command is mainly a convenience command that
eases typesetting such lists of values separated by some token (a comma by
default).  As its name suggests, it also comes with higher order separators
which are dots by default.  More interrestingly, it also allows to apply a
macro to each element. Hence the second example given above was produced using:

\begin{verbatim}
$\{\lip |\mathbf{h}(A_#1^2)| {1, 2,, n}$
\end{verbatim}

Note that the command makes use of some argument \verb!#1!, and that nesting calls to \texttt{\textbackslash sip} needs some extra care. Each inner level needs doubling the number of hash signs, and extra braces are needed so that pipes are parsed correctly. For example:

\[\lip[sep=\cup] |{\{\lip[sep={,}] |a_##1| {1, 2,, #1}\}}| {5, 6, 7}\]

\begin{verbatim}
\[\lip[sep=\cup] |{\{\lip[sep={,}] |a_##1| {1, 2,, #1}\}}| {5, 6, 7}\]
\end{verbatim}

\section{The \texttt{\textbackslash lip} command}

In its simplest form, \texttt{\textbackslash lip} takes a list of
comma-separated values. Each empty value will be replaced with an ellipsis (by
default dots, see Section~\ref{sec:lip}).

\example{$\lip{1, 2, 3}$}

\example{$\lip{1, 2,, n}$}

\example{$\lip{a, b,, g, h,, z}$}

\example{$\lip{, 1, 2,, n}$}

\example{$\lip{1, 2,, n,}$}

\example{$\lip{, -1, 0, 1, }$}

One may use the \texttt{\textbackslash lip} command in text mode.

\example{\lip{A, B, C}}

It is also possible to provide a command that will transform each of the
values. The command has to be given just before the values, between two
vertical bars. This command should use the \verb!#1!  as the value being
stylized.

\begin{center}
  \begin{tabular}{p{0.42\textwidth} p{0.42\textwidth}}
    \toprule
    \customtt{$\lip |A_#1| {1, 2,, n}} & $\lip |A_#1| {1, 2,, n}$ \\
    \bottomrule
  \end{tabular}
\end{center}

\example{$\lip |X| {1, 2, 3}$}

\section{The options}

The following options may be set for the whole \TeX\ group using the
\texttt{\textbackslash lipsetdefault} command, or locally within the
\texttt{\textbackslash lip} command, as shown in the examples.

\subsection{allowbreak (= true)}

You can disable this feature using \texttt{\textbackslash
lipsetdefault{allowbreak=false}} and then use \texttt{\textbackslash
lip[allowbreak]} when you want ellipses to be breakable.

\example{$\lip[allowbreak=false] |longlongthing| {1, 2, 3, 4, 5}$}

\example{$\lip[allowbreak] |longlongthing| {1, 2, 3, 4, 5}$}

\subsection{sep (= \texttt{,})}

Sets the separator between values.

\example{$\lip[sep=\cup] {A, B,, K}$}

\subsection{lip (= \texttt{\dots})} \label{sec:lip}

Sets the ellipsis command(s).

\example{$\lip[lip=\cdots] {a, b,, c, d}$}

\end{document}
